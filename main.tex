\documentclass[11pt,a4paper]{article}
\usepackage[spanish]{babel}
\usepackage[utf8]{inputenc}
\usepackage{amsmath}
\usepackage{amsfonts}
\usepackage{amssymb}
\usepackage{blindtext}

\usepackage{relsize}

\usepackage{hyperref}
 \hypersetup{
     colorlinks=true,
     linkcolor=black,
     filecolor=blue,
     citecolor = black,      
     urlcolor=blue,
     }

\usepackage{enumitem}
  
\usepackage[toc,page]{appendix}

\usepackage{ucs}
\usepackage{listings}
\usepackage{geometry}
\geometry{
	margin = 1in
}

\author{
	\begin{tabular}{rl}
		Fernández González, Álvaro. & DNI 50357343Q\\
		Fernández González, Gonzalo. & DNI 50357342S\\
		López Gómez, Alejandro. & DNI 48207387P\\
		Sánchez Diaz, Jorge. & DNI 50252083G \\
		Siodmok, Oskar Denis. & NIE Y0215268-W 
	\end{tabular}
}

\title{Comparación del Código Ético de la Universidad Rey Juan Carlos y del Códgio Deontológico de la Profesión Farmacéutica\\
	[0.3em]\large Trabajo de la asignatura de Principios Jurídicos de la ETSII

	} 
\begin{document}
\maketitle
\tableofcontents\pagebreak

%%%% Paragraph formating
\setlength\parindent{.5in}
%\setlength{\parskip}{0.175\baselineskip}%
%%%%

\section{Introducción}
\subsection{¿Por qué estos dos códigos?}
Para elegir un 2º código con el que comparar el CEURJC\footnote{Por cuestiones de simpleza y limpieza nos referimos al al Código Ético de la Universidad Rey Juan Carlos como CEURJC y al Código Deontológico de la Profesión Farmacéutica como CDF.}, quedamos un día y nos reunimos para proponer ideas.

Primero pensamos que sería buena idea comparar los códigos de dos universidades. A ver que diferencias éticas podía haber entre dos instituciones con la misma finalidad. Buscamos códigos éticos de otras universidades como el de la UPM, pero al final vimos que realmente no daba mucho contenido y la descartamos.

Tras esto pensamos que para hacer una comparación interesante, necesitábamos un código interesante. Así que miramos trabajos en los cuales tener valores y aplicarlos a dilemas morales estuviese a la orden del día.

Miramos códigos penitenciarios y nos parecieron que se correspondían con lo que estábamos buscando. Sin embargo, uno de nosotros propuso el código ético de la profesión farmacéutica pues su padre trabajaba en ello y si teníamos dudas nos lo podría desarrollar y poner ejemplos. Valoramos y escogimos este último.

\subsection{Hipótesis}
%REDACCIÓN ORIGINAL DE RISI
%Habiendo visto únicamente el Código de la URJC, imagino que el de los farmacéuticos será menos general y manejará conceptos menos abstractos con el fin de detallar más claramente ciertos límites en la profesión.
%La brevedad del CEURJC me hace pensar que está diseñado con un fin estético de cara al público más que de buscar unos criterios éticos necesarios para la institución como el CEF.

A primera vista lo que más llama la atención es la diferencia de extensión entre los códigos. Así, la brevedad del CEURJC nos hizo pensar que el propósito de este podría ser estético de cara al público y no buscar unos criterios éticos necesarios para la institución, como ocurriría en el CDF. Por otro lado, supusimos que el CDF tendría un carácter menos general y desarrollaría conceptos menos abstractos con el fin de detallar con claridad los límites de la profesión. Y no solo habría una diferencia en la generalidad de los códigos, sino que también una diferencia en la explicitud. Tras una lectura superficial del CEURJC nuestras predicciones sobre las diferencias entre los códigos se colmaron de matices.
	
Primero, nos pareció que el CDF no le daría tanta prioridad a cuestiones como la argumentación racional y al rigor, como se hace en el punto ``RESPETO Y COLABORACIÓN'' del CEURJC. Lo mismo pensamos de cuestiones como la colaboración o la investigación, que tienen más sentido en un contexto universitario y no profesional. En cambio, sobre cuestiones básicas y normalizadas en toda sociedad moderna, hipotetizamos que sí habría similitudes claras. Esto es, la defensa de la igualdad independientemente de las ideas y las condiciones y la defensa de la libertad a la hora de ejercer derechos fundamentales.

Hay un punto de especial interés en el CEURJC, este es el de ``PARTICIPACIÓN DEMOCRÁTICA''. Realmente no sabemos como es el contexto de la profesión farmacéutica. Además, el CDF parece específico para la profesión misma. Es por eso que dedujimos que el CDF no tendría ningún tipo de explicación respecto a participaciones democráticas y procesos electorales, pues estos solo tendrían sentido en un contexto de ``microsociedad'' como lo es la universidad. Lo mismo pasa en el punto ``BUEN GOBIERNO, TRANSPARENCIA Y RENDICIÓN DE CUENTAS'' del CEURJC: cuestiones como la ``transparencia'' o ``justicia'' de un Gobierno solo tienen sentido si tal Gobierno existe.

Respecto a todo lo que es la imagen pública tanto de universitarios como farmacéuticos, las cuestiones de compromiso con la institución del CEURJC seguramente tengan su correspondencia en el CDF. Puede que tal correspondencia no busque la mejora de la institución sino la mejora de la comunidad farmacéutica y su imagen como un todo. Lo mismo seguramente ocurra con el énfasis que pone el CEURJC a la calidad de los servicios que ofrece la universidad y el respeto a la sociedad por parte de la institución y los universitarios. En ese sentido, probablemente el CDF será semejante en buscar una calidad en los medicamentos fabricados, investigados y vendidos para ayudar al máximo números de personas y a la sociedad. 

Finalmente, es probable que el CDF también de importancia a la responsabilidad y asunción de consecuencias pues, al fin y al cabo, el ámbito farmacéutico tiene mucha más seriedad que el universitario. En lo que respecta a la difusión de conocimientos científicos, da la impresión de que esta solo tiene sentido en la esfera de la educación y la universidad.

\section{Diferencias entre los códigos}
\subsection{Generalidades y ámbitos de aplicación}
El CDF tiene un carácter general, para cualquier farmacéutico, a diferencia del CEURJC, el cual es específico para la universidad. Este carácter general, o de más bajo nivel, tiene prioridad (en España) ante cualquier otro código y sirve como base para construir códigos más específicos dentro de las diferentes modalidades de la profesión. Por otro lado, el código de la URJC es justamente un ejemplo de un código ético específico. 

Este ámbito de aplicación necesita de unos límites claros. Así, el CEURJC solo se aplicaría dentro de la ``microsociedad'' que es la universidad mientras que el CDF sería de obligado cumplimiento para cualquier profesional farmacéutico que ejerza en España, sea español o europeo. Ya que este código farmacéutico tiene tal carácter general y básico, no solo aplicaría a los profesionales, sino que a las Sociedades que estos creen, Sociedades que responderían deontológicamente a las acciones de sus integrantes. 

\subsection{Principios básicos}
 	Como institución pública, la universidad se compromete a ofrecer buenos servicios a la sociedad respecto a la calidad en la enseñanza, la difusión de conocimientos científicos y la buena administración. En contraste, el CDF claramente especifica en mayor profundidad los principios de compromiso con la sociedad. Esto es debido a que la vida de los españoles está en manos de profesionales como los farmacéuticos. Por lo tanto, el código será estricto respecto al trato a los pacientes. Esto es, tratar a los pacientes de forma igualitaria, anteponer su salud sobre los intereses personales, etc. 

La universidad, como reflejo de la sociedad que es, enfatiza en las relaciones sociales. La argumentación lógica, el respeto y la justicia son prácticas necesarias para la colaboración. Estas nociones también son de importancia para los farmacéuticos, pero no tanto como lo es, otra vez, el servicio a los pacientes. Así, mientras que los universitarios tendrán que colaborar por la investigación y transmisión de conocimientos científicos, los farmacéuticos cooperarán por una sociedad más saludable y mejor atendida, dando gran importancia al derecho a la salud. 

Según el CDF, será de gran importancia que el profesional se mantenga actualizado respecto a conocimientos científicos y legales. Además, tendrá que tener cuidado sobre lo que es y no capaz de hacer. Dichos puntos no tienen un reflejo claro en el código universitario, en virtud de la falta de necesidad de ellos, pues al fin y al cabo, las vidas de las personas no dependen de los conocimientos científicos ni legales de los universitarios. 

 	El farmacéutico no puede encubrir ninguna acción legal. Si se da algún acto ilegal el farmacéutico tendrá que comunicarlo con las autoridades y el colegio farmacéutico correspondiente. La ley es de extrema importancia para el farmacéutico. Otra vez y debido a la diferencia en seriedad de los casos, el CEURJC no enfatiza tanto en cuestiones legales. Lo mismo pasa con las titulaciones. Para ejercer de farmacéutico hace falta probar unos conocimientos, obviamente. En esto se diferencia la universidad, pues es pública y universal; al fin y al cabo el conocimiento debería de estar al alcance de todos. 

De la misma forma que los productos ofrecidos por los farmacéuticos serán eficaces con una eficacia probada en rigor, el código de la URJC establece que el rigor tiene que estar no solo en los servicios sino en la transferencia de conocimientos.	

\subsection{Objetividad y subjetividad}
Mientras que la universidad abre la puerta a la diversidad de opiniones y al debate de estas, en el campo farmacéutico y en toda ciencia no existe la subjetividad. Es por eso que por encima de toda valoración subjetiva del farmacéutico tiene que estar el rigor científico y los hechos legales. Rigor y leyes que el profesional en cuestión proporcionará y antepondrá a sus intereses personales. 

Aun así, la ética y deontología del farmacéutico tiene importancia ante la ley. Si una nueva ley obliga al profesional a actuar en contra de sus valores podrá presentar una queja al colegio farmacéutico. El farmacéutico tiene que ser claro con sus pacientes, no debe de aceptar sobornos y debe de conocer las consecuencias de sus actos. En otras palabras, responderá personalmente a los daños que cause y aceptará los efectos negativos de estos. El código ético de la URJC tiene su reflejo a estos puntos: el universitario tiene que buscar la difusión científica y fáctica en sociedad, ser responsable y asumir las consecuencias de sus acciones y errores. En el ámbito farmacéutico, un error del profesional puede ser sancionado con indemnizaciones mientras que en la universidad las consecuencias no llegan a ser tan extremas (pues no es requerido de ello), pero aun así se promueve la responsabilidad al actuar.

 










\section{Conclusiones}



\pagebreak
%**\appendix
\setcounter{secnumdepth}{0}
\section{Anexos}
%\section{Código Deontológico de la Profesión Farmacéutica}
%\label{what}
\begin{itemize}[label = {}]
	\item Anexo 1. \href{https://www.portalfarma.com/Profesionales/organizacionfcolegial/portal-transparencia/Documents/2018-Codigo-Deontologia-Profesion-Farmaceutica-CGCOF.pdf}{Código Deontológico de la Profesión Farmacéutica}.
	\item Anexo 2. \href{https://www.urjc.es/codigoetico}{Código Ético de la Universidad Rey Juan Carlos}.
\item Anexo 3. Código \href{http://www.github.com/DennisDv24}{\LaTeX} del trabajo.
\item Anexo 4. \href{https://docs.google.com/document/d/1JPIhGiP2SUPTwZ42QgbEhY5bHKPIPoN0i3u79ej2DwQ/edit#heading=h.zf8tz9rpog2d}{Borrador del trabajo}.
\item Anexo 5. \href{https://docs.google.com/document/d/1jByW4G1lbvmoq82MXhN4lJDx3bpDZP1n1jSouQVpACU/edit#heading=h.ocedmy22huli}{Borrador de los resúmenes del Código Deontológico de la Profesión Farmacéutica}.
\end{itemize}




\end{document}
